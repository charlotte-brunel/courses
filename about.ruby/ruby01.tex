\documentclass{article}
\usepackage[utf8]{inputenc}
\usepackage{hyperref}
\usepackage{graphicx}
\usepackage{listings}
\usepackage{amssymb}
\usepackage{color}
\usepackage{xcolor}
\usepackage{indentfirst}
\usepackage{makeidx}
\addtolength{\oddsidemargin}{-.875in}
\addtolength{\evensidemargin}{-.875in}
\addtolength{\textwidth}{1.75in}

\makeindex

\newcommand{\ruby}{\huge{Ruby}}

\newcommand{\csection}{\huge{C}.\\}

\lstset{language=ruby,frame=single,backgroundcolor=\color{yellow},numbers=left,breaklines=true,breakautoindent=true,basicstyle=\small}
\date{\today}
\title{Ruby, my notebook.\\\includegraphics[scale=1.0,width=64px,height=64px]{ruby64.png}}
\author{Pierre Lindenbaum\\\href{https://twitter.com/yokofakun}{@yokofakun}\\\url{http://plindenbaum.blogspot.com} }



\begin{document}
\maketitle

\tableofcontents

\section{General}

\begin{tabular}{ r || l }
Java & Ruby \\
\hline                        
this & self \\
null & nil \\
.getClass & class \\
Integer.parseInt(s) & Integer(s) \\
\end{tabular}

\subsection{Global variable}
\begin{lstlisting}
$var1 = "Hello"
\end{lstlisting}
\subsection{Instance variable}
\begin{lstlisting}
@var2 = "World"
\end{lstlisting}

\subsection{Class variable}
\begin{lstlisting}
@@var3 = "blabala"
\end{lstlisting}

\subsection{Numbers}
\subsubsection{times}
\begin{lstlisting}
10.times { print "CA" }
=>CACACACACACACACACACA
\end{lstlisting}
\subsubsection{upto}
\begin{lstlisting}
5.upto(7) {|i| puts "i=#{i};"}
i=5;
i=6;
i=7;
\end{lstlisting}

downto
\begin{lstlisting}
100.downto(98) {|i| puts "i=#{i};"}
i=100;
i=99;
i=98;
\end{lstlisting}

step
\begin{lstlisting}
1.step(10,2) {|i| puts "i=#{i};"}
i=1;
i=3;
i=5;
i=7;
i=9;
\end{lstlisting}

\begin{lstlisting}
99.downto(95) {|i| puts "i=#{i};"}
i=99;
i=98;
i=97;
i=96;
i=95;
\end{lstlisting}



\subsection{String}
shortcuts
\begin{lstlisting}
$var1 = "Hello"
puts "#$var1, World!"
\end{lstlisting}

\begin{lstlisting}
puts %q/hello world with ' and " /
puts %Q/hello world with ' and " /

=>
hello world with ' and " 
hello world with ' and " 
\end{lstlisting}

here document
\begin{lstlisting}
puts <<STOP_HERE
You say "Good Bye"
I say 'hello'
STOP_HERE
=>
You say "Good Bye"
I say 'hello'
\end{lstlisting}

chomp
\begin{lstlisting}
"hello".chomp            #=> "hello"
"hello\n".chomp          #=> "hello"
"hello\r\n".chomp        #=> "hello"
\end{lstlisting}
chop
\begin{lstlisting}
"string\r\n".chop   #=> "string"
"string\n\r".chop   #=> "string\n"
"string\n".chop     #=> "string"
"string".chop       #=> "strin"
"x".chop.chop       #=> ""
\end{lstlisting}
chr
\begin{lstlisting}
a = "abcde"
a.chr    #=> "a"
\end{lstlisting}
clear
\begin{lstlisting}
a = "abcde"
a.clear    #=> ""
\end{lstlisting}
split
\begin{lstlisting}
name,site = "EcoRI,GAATTC".split(/[,]/)
puts "#{name}=#{site}" #=> EcoRI=GAATTC
\end{lstlisting}

\subsection{Function}
\begin{lstlisting}
def say_hello(name)
	result = "Hello, " + name
	return result
end
puts say_hello("World")
\end{lstlisting}
or
\begin{lstlisting}
$var1 = "Hello"
def say_hello(name)
	result = "#$var1, #{name}"
	return result
end
puts say_hello("World")
\end{lstlisting}



\subsection{If-Then-Else}
\begin{lstlisting}
enz="EcoRI"

if enz=="BamHI"
	print "GGATCC"
elsif enz=="EcoRI"
	print "GAATTC"
else
	puts "Another enzyme"
end

=> GAATTC=> nil
\end{lstlisting}

\subsection{While loop}
\begin{lstlisting}
seq="";
while seq.length < 100
	seq += "A"
end
print seq
AAAAAAAAAAAAAAAAAAAAAAAAAAAAAAAAAAAAAAAAAAAAAAAAAAAAAAAAAAAAAAAAAAAAAAAAAAAAAAAAAAAAAAAAAAAAAAAAAAAA=> nil
\end{lstlisting}
or
\begin{lstlisting}
seq=""
seq += "A"   while seq.length < 100
puts seq
AAAAAAAAAAAAAAAAAAAAAAAAAAAAAAAAAAAAAAAAAAAAAAAAAAAAAAAAAAAAAAAAAAAAAAAAAAAAAAAAAAAAAAAAAAAAAAAAAAAA
=> nil
\end{lstlisting}

\subsection{Regex}
test regex
\begin{lstlisting}
s="ATGACTACGATC"

if s =~ /[ATGC]+/
	puts "Acid Nucleic"
else
	puts "Not a DNA"
end

=> Acid Nucleic
\end{lstlisting}

replace first occurence of regex:
\begin{lstlisting}
puts "GAATCGTACGATCGCTAGA".sub(/T/,"U")
=> GAAUCGTACGATCGCTAGA
\end{lstlisting}
replace all occurences of regex:
\begin{lstlisting}
puts "GAATCGTACGATCGCTAGA".gsub(/T/,"U")
=> GAAUCGUACGAUCGCUAGA
\end{lstlisting}
\subsection{Yield statement}

\begin{lstlisting}
def print_seq
	puts "<Sequence>"
	yield
	puts "</Sequence>"
end
print_seq { puts "GAAATTTATAC" }

=> <Sequence>
GAAATTTATAC
</Sequence>
\end{lstlisting}

\begin{lstlisting}
def print_enz
	puts "<enzymes>"
	yield("EcoRI","GAATTC")
	yield("BamHI","GGATCC")
	puts "</enzymes>"
end
print_enz {|name,site| puts "<site name=\"#{name}\">#{site}</site>" }
print_enz {|name,site| puts "<enzyme><name>#{name}</name><site>#{site}</site></enzyme>" }

=> <enzymes>
<site name="EcoRI">GAATTC</site>
<site name="BamHI">GGATCC</site>
</enzymes>

<enzymes>
<enzyme><name>EcoRI</name><site>GAATTC</site></enzyme>
<enzyme><name>BamHI</name><site>GGATCC</site></enzyme>
</enzymes>
\end{lstlisting}

\subsubsection{ 'each'}
\begin{lstlisting}
%w( EcorRI  BamHI PstI NotI).each {|name| puts name }
=>
EcorRI
BamHI
PstI
NotI
\end{lstlisting}

\subsection{IO}
\subsubsection{Reading}
\begin{lstlisting}
line=gets
puts line
\end{lstlisting}

\subsection{Class}
\begin{lstlisting}
class GeneticCode
	def initialize(name,ncbiString)
		@name=name
		@ncbiString=ncbiString
	end
end

std = GeneticCode.new(
	"standard",
	"FFLLSSSSYY**CC*WLLLLPPPPHHQQRRRRIIIMTTTTNNKKSSRRVVVVAAAADDEEGGGG"
	)
\end{lstlisting}
\subsubsection{inspect}
\begin{lstlisting}
std = GeneticCode.new(
	"standard",
	"FFLLSSSSYY**CC*WLLLLPPPPHHQQRRRRIIIMTTTTNNKKSSRRVVVVAAAADDEEGGGG"
	)

puts std.inspect
=>#<GeneticCode:0xb77b0a04 @ncbiString="FFLLSSSSYY**CC*WLLLLPPPPHHQQRRRRIIIMTTTTNNKKSSRRVVVVAAAADDEEGGGG", @name="standard">
\end{lstlisting}


\subsubsection{to\_s}
\begin{lstlisting}
class GeneticCode
	def initialize(name,ncbiString)
		@name=name
		@ncbiString=ncbiString
	end
	
	def to_s
		"Genetic code #@name"
	end
end

std = GeneticCode.new(
	"standard",
	"FFLLSSSSYY**CC*WLLLLPPPPHHQQRRRRIIIMTTTTNNKKSSRRVVVVAAAADDEEGGGG"
	)

puts std

=> "Genetic code standard"
\end{lstlisting}
\subsubsection{Sub-Classing}
\begin{lstlisting}
class GeneticCode
	def initialize(name,ncbiString)
		@name=name
		@ncbiString=ncbiString
	end
	
	def to_s
		"Genetic code #@name"
	end
end

class MitochondrialCode < GeneticCode
	def initialize()
		super("Mitochondrial",
		"FFLLSSSSYY**CCWWLLLLPPPPHHQQRRRRIIMMTTTTNNKKSS**VVVVAAAADDEEGGGG"
		)
		@url="http://www.ncbi.nlm.nih.gov/Taxonomy/Utils/wprintgc.cgi#SG2"
	end
end
g = MitochondrialCode.new

puts g
=> "Genetic code Mitochondrial"
\end{lstlisting}
\subsubsection{Calling 'super' in subclass}
\begin{lstlisting}
(...)
class MitochondrialCode < GeneticCode
(...)
	def to_s
		"I am "+ super
	end
end
g = MitochondrialCode.new
puts g
=>I am Genetic code Mitochondrial
\end{lstlisting}

\subsubsection{accessing fields}
\begin{lstlisting}
class GeneticCode
	def initialize(ncbiString)
		@ncbiString=ncbiString
	end
end

g = GeneticCode.new("FFLLSSSSYY**CC*WLLLLPPPPHHQQRRRRIIIMTTTTNNKKSSRRVVVVAAAADDEEGGGG")

puts g.ncbiString
=> -:9: undefined method `ncbiString' for #<GeneticCode:0xb7461ad8> (NoMethodError)
\end{lstlisting}
now, define a getter:

\begin{lstlisting}
class GeneticCode
	def initialize(ncbiString)
		@ncbiString=ncbiString
	end
	
	def ncbiString
		@ncbiString
	end 
end
g = GeneticCode.new("FFLLSSSSYY**CC*WLLLLPPPPHHQQRRRRIIIMTTTTNNKKSSRRVVVVAAAADDEEGGGG")
puts g.ncbiString
=> FFLLSSSSYY**CC*WLLLLPPPPHHQQRRRRIIIMTTTTNNKKSSRRVVVVAAAADDEEGGGG
\end{lstlisting}

shortcut with 'attr\_reader'
\begin{lstlisting}
class GeneticCode
	attr_reader :name, :ncbiString

	def initialize(name,ncbiString)
		@name=name
		@ncbiString=ncbiString
	end
end

g = GeneticCode.new("Standard","FFLLSSSSYY**CC*WLLLLPPPPHHQQRRRRIIIMTTTTNNKKSSRRVVVVAAAADDEEGGGG")
puts g.ncbiString
=>FFLLSSSSYY**CC*WLLLLPPPPHHQQRRRRIIIMTTTTNNKKSSRRVVVVAAAADDEEGGGG
\end{lstlisting}


defining a setter:
\begin{lstlisting}
class Sample
	def name
		@name
	end
	
	def name=(name)
		@name=name
	end
end
s = Sample.new
s.name="Blood01"
puts "Sample name is #{s.name} "
=> Sample name is Blood01
\end{lstlisting}
\subsubsection{static class variables}
\begin{lstlisting}
class Sample
	@@id_generator=0
	def initialize
		## @@id_generator++ doesn't work (?)
		@@id_generator+=1
		@id=@@id_generator
	end
	
	def id
		@id
	end
end
s = Sample.new
puts "Sample ID is #{s.id} "
s = Sample.new
puts "Sample ID is #{s.id} "

=>
Sample ID is 1 
Sample ID is 2 
\end{lstlisting}

\subsubsection{static class Method}
\begin{lstlisting}
class Sample
	@@id_generator=0
	
	def Sample.newId
		@@id_generator+=1
	end
	
	def initialize
		@id=Sample.newId
	end
	
	def id
		@id
	end
end
s = Sample.new
puts "Sample ID is #{s.id} "
s = Sample.new
puts "Sample ID is #{s.id} "

puts "Another ID is #{Sample.newId} "

=>
Sample ID is 1 
Sample ID is 2 
Another ID is 3 
\end{lstlisting}

\subsubsection{final / constant field}
\begin{lstlisting}
class Sample
	DEFAULT_ID_GENERATOR=100
	@@id_generator=DEFAULT_ID_GENERATOR
	
	def Sample.newId
		@@id_generator+=1
		##DEFAULT_ID_GENERATOR=0 would fail:  dynamic constant assignment
	end
	
	def initialize
		@id=Sample.newId
	end
	
	def id
		@id
	end
end
s = Sample.new
puts "Sample ID is #{s.id} "
s = Sample.new
puts "Sample ID is #{s.id} "

puts "Another ID is #{Sample.newId} "

=>
Sample ID is 101 
Sample ID is 102 
Another ID is 103 
\end{lstlisting}

\subsubsection{Private Constructor with private\_class\_method}
\begin{lstlisting}
class Genotype
	private_class_method :new
	attr_reader :a1,:a2
	
	def initialize(a,b)
		@a1=(a<b ? a : b)
		@a2=(a<b ? b : a)
	end
	
	def Genotype.newInstance(a,b)
		new(a,b)
	end
end
g = Genotype.newInstance("C","A")
puts "#{g.a1}/#{g.a2} "

## g= Genotype.new("C","A")  <- FAIL private method `new' called for Genotype:Class (NoMethodError)

=>
A/C
\end{lstlisting}

\subsubsection{private/protected/public}
\begin{lstlisting}
class A
	public
	
		def f1
			1
		end
	
	protected
	
		def f2
			2
		end
		
	private
	
		def f3
	
		end
end


i = A.new
puts i.f1  #1
# puts i.f2 FAILS protected method `f2' called for #<A1:0xb73f5798> (NoMethodError)
# puts i.f3 FAILS private method `f3' called for #<A1:0xb73a2750> (NoMethodError)

class B < A
	public
		def f4
			f2
		end
	
	
		def f5
			f3
		end
end

i = B.new
puts i.f4 # 2
puts i.f5 # nil!
\end{lstlisting}
or using methods:
\begin{lstlisting}
class A
	def f1
		1
	end

	def f2
		2
	end

	def f3

	end

	public :f1
	protected :f2
	private :f3
end


i = A.new
puts i.f1 # => 1
puts i.f2 # => protected method `f2' called for #<A:0xb743c698> (NoMethodError)
\end{lstlisting}

\subsubsection{Freezing objects}
\begin{lstlisting}
class Sample
	def name
		@name
	end
	
	def name=(name)
		@name=name
	end
end
s = Sample.new
s.name="Blood01"

puts "Sample name is #{s.name} "

s.freeze
s.name="Blood02" => ERROR can't modify frozen object (TypeError)
\end{lstlisting}

\subsection{Collections}

\subsubsection{Arrays}

\begin{lstlisting}
a = [ TRUE, 'GAATTC', "GGATCC", 0.5]
a[0]
=> true
> puts "EcorRI cuts #{a[1]}"
=> EcorRI cuts GAATTC
a[3] = nil
a
=> [true, "GAATTC", "GGATCC", nil]
\end{lstlisting}

or using '\%w'
\begin{lstlisting}
 %w{ EcoRI BamHI PstI NotchI }
=> ["EcoRI", "BamHI", "PstI", "NotI"]
\end{lstlisting}

or using Array.new
\begin{lstlisting}
a=Array.new
a[0]="EcorRI"
a[10]="BamH1"

puts a
=> 

EcorRI
nil
nil
nil
nil
nil
nil
nil
nil
nil
BamH1
\end{lstlisting}

length:
\begin{lstlisting}
a=Array.new
a[0]="EcorRI"
a[1]="BamH1"
a[2]="Pst1"
puts a.length # => 3
a[10]="NotI"
puts a.length # => 11
\end{lstlisting}
negative index:
\begin{lstlisting}
a=Array.new
a[0]="EcorRI"
a[1]="BamH1"
a[2]="Pst1"
puts a[-1]
=>PstI
\end{lstlisting}

sub-list with [ index , count ]
\begin{lstlisting}
a = %w{ PstI EcoRI BamHI NotI HindIII SmaI XbaI }
puts a[3,2]
=>  NotI HindIII
puts a[-5,2]
=> BamHI NotI
\end{lstlisting}

sub-list with [ start..end ] (inclusive) or [ start...end ] (exclusive)
\begin{lstlisting}
a = %w{ PstI EcoRI BamHI NotI HindIII SmaI XbaI }
puts a[0..3] #=> PstI EcoRI BamHI NotI 
puts a[0...3] #=> PstI EcoRI BamHI 
puts a[1..1]  #=> EcoRI
puts a[1...1] #=> (empty)
puts a[-1..2]  #=> (empty)
puts a[-3..-2] #=> HindIII SmaI
\end{lstlisting}

\begin{tabular}{ r || l }
Java & Ruby \\
\hline                        
add & push \\
length & length \\
removeLast & pop \\
removeFirst & shift \\
get(index) & [index] \\
\end{tabular}

Iterations on arrays:
\begin{lstlisting}
$a = %w{ PstI EcoRI BamHI NotI HindIII SmaI XbaI }

def findEnzymeIndexByName(name)
	for i in 0 ... $a.length
		return i if($a[i]==name) 
	end
	-1;
end

puts findEnzymeIndexByName("SmaI") #=> 5
puts findEnzymeIndexByName("zzzzz") #=> -1
\end{lstlisting}


with blocks
\begin{lstlisting}
$a = %w{ PstI EcoRI BamHI NotI HindIII SmaI XbaI }

def findEnzymeIndexByRegex(pattern)
	$a.find {|enzname| enzname =~ pattern }
end

puts findEnzymeIndexByRegex(/.*II/) #=> HindIII
puts findEnzymeIndexByRegex(/.*V/)  #=> nil
\end{lstlisting}

collect

\begin{lstlisting}
a = %w{ PstI EcoRI BamHI NotI HindIII SmaI XbaI }

def findEnzymeIndexByRegex(pattern)
	$a.find {|enzname| enzname =~ pattern }
end

puts a.collect {|enzname| enzname =~ /aI/ }
=>
nil
nil
nil
nil
nil
2
2
\end{lstlisting}

inject
\begin{lstlisting}
a = %w{ PstI EcoRI BamHI NotI HindIII SmaI XbaI }
puts a.inject("LIST IS ") {|s,enzname| s+= " [#{enzname}]" }
#=>LIST IS  [PstI] [EcoRI] [BamHI] [NotI] [HindIII] [SmaI] [XbaI]
\end{lstlisting}





\subsubsection{Associative Arrays}
\begin{lstlisting}
> map={ 'EcoRI' => 'GAATTC', 'BamHI' => 'GGATACC', 'PstI' => 'CGATCG' }
=> {"EcoRI"=>"GAATTC", "BamHI"=>"GGATACC", "PstI"=>"CGATCG"}

> puts map["EcoRI"]
=> GAATTC

> puts "Enzyme is #{map['EcoRI']}"
=> Enzyme is GAATTC
\end{lstlisting}

or

\begin{lstlisting}
> m=Hash.new()
=> {}
> m['EcoRI']="GAATTC";
> puts m['BamHI']
=> nil
> puts m['EcoRI']
GAATTC
\end{lstlisting}

length
\begin{lstlisting}
m={ 'EcoRI' => 'GAATTC', 'BamHI' => 'GGATACC', 'PstI' => 'CGATCG' }
puts m.length #  => 3
\end{lstlisting}


foreach
\begin{lstlisting}
m ={ 'EcoRI' => 'GAATTC', 'BamHI' => 'GGATACC', 'PstI' => 'CGATCG' }
m.each {|name,site| puts "#{name} cuts #{site}"}
BamHI cuts GGATACC
EcoRI cuts GAATTC
PstI cuts CGATCG
\end{lstlisting}

\subsection{I/O}
\begin{lstlisting}
f = File.open("jeter.rb")
f.each do |s|
puts s
end
f.close
=> (print self)
\end{lstlisting}

\subsection{Blocks}
\begin{lstlisting}
class EnzymeLabel
	def initialize(name,&fun)
		@name=name
		@fun=fun
	end
	
	def label
		@fun.call(self)
	end
	
end

label=EnzymeLabel.new("EcoRI") { " cuts GAATTC"  }
puts label.label #=>  cuts GAATTC

label=EnzymeLabel.new("EcoRI") { " coupe G^AATTC"  }
puts label.label #=>  coupe G^AATTC
\end{lstlisting}


\end{document}



